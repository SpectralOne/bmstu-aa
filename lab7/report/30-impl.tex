\chapter{Технологическая часть}

В данном разделе приведены требования к программному обеспечению, средства реализации и листинги кода.

\section{Требования к ПО}

К программе предъявляется ряд требований:
\begin{itemize}
	\item корректная сортировка.
\end{itemize}

\section{Средства реализации}

В качестве языка программирования для реализации данной лабораторной работы был выбран многопоточный язык GO \cite{golang}. Данный выбор обусловлен моим желанием расширить свои знания в области применения данного язкыа. Так же данный язык предоставляет средства тестирования разработанного ПО. Так же была использована библиотека {\ttfamily gofakeit} \cite{faker} для генерации случайных записей.

\clearpage

\section{Листинг кода}

В листингах \ref{lst:brute} -- \ref{lst:combined} приведены реализации алгоритмов.

\begin{lstinputlisting}[
	caption={Алгоритм полного перебора},
	label={lst:brute},
	style={go},
	linerange={81-90},
	]{../src/dict/dict.go}
\end{lstinputlisting}

\begin{lstinputlisting}[
	caption={Реализация частотного анализа},
	label={lst:fa},
	style={go},
	linerange={109-146},
	]{../src/dict/dict.go}
\end{lstinputlisting}

\begin{lstinputlisting}[
	caption={Реализация алгоритма эффективного поиска},
	label={lst:combined},
	style={go},
	linerange={149-159},
	]{../src/dict/dict.go}
\end{lstinputlisting}

В таблице \ref{tabular:functional_test} приведены функциональные тесты программы.

Используемый в тестах словарь:

Dict = 
\begin{tabular}{l l l}
	[\{name : apple, & email : apple@apple.com, & city : coupertino \}\\
	\{name : microsoft, & email : m@msn.com, & city : washington \}]
\end{tabular}

\renewcommand{\arraystretch}{2}
\begin{table}[h]
	\begin{center}
		\caption{Функциональные тесты}
		\label{tabular:functional_test}
		\begin{tabular}{|*3{>{\renewcommand{\arraystretch}{1}}c|}}
			\hline
			\textbf{Описание теста} & \textbf{Входные данные} & \textbf{Ожидаемый результат}\\
			\hline
			\text{успешный поиск} & $\begin{array}{cc} Dict & apple\end{array}$ &
			$\begin{array}{ccc} ``name`` &:& ``apple``\\ ``email`` & :& ``apple@apple.com`` \\ ``city``&:&``coupertino``\end{array}$\\
			\hline
			\text{неуспешный поиск} & $\begin{array}{cc} Dict & xiaomi\end{array}$ & \text{Not found}\\
			\hline
			
		\end{tabular}
	\end{center}
\end{table}

\section*{Вывод}

В данном разделе была рассмотрена структура ПО и листинги кода программы.
