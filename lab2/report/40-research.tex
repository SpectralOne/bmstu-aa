\chapter{Исследовательская часть}

\section{Пример работы}

Демонстрация работы программы приведена на рисунке \ref{img:mult_demo}.

\boximg{160mm}{mult_demo}{Демонстрация работы алгоритмов умножения матриц}

\section{Технические характеристики}

Технические характеристики устройства, на котором выполнялось тестирование:

\begin{itemize}
	\item Операционная система: Kali \cite{kali} Linux \cite{linux} 5.8.10-1kali1 64-bit.
	\item Память: 8 GB.
	\item Процессор: Intel® Core™ i5-8250U \cite{intel} CPU @ 1.60GHz
\end{itemize}

Тестирование проводилось на ноутбуке при включённом режиме производительности. Во время тестирования ноутбук был нагружен только системными процессами.

\section{Время выполнения алгоритмов}

Алгоритмы тестировались при помощи написания <<бенчмарков>> \cite{gotest}, предоставляемых встроенными в Go средствами. Для такого рода тестирования не нужно самостоятельно указывать количество повторов. В библиотеке для тестирования существует константа $N$, которая динамически варьируется в зависимости от того, был ли получен стабильный результат или нет.

В листинге \ref{lst:mult_test} пример реализации бенчмарка.

\begin{lstinputlisting}[
	caption={Реализация бенчмарка},
	label={lst:mult_test},
	style={go},
	linerange={1-17},
	]{../src/multiplication/multiplication_test.go}
\end{lstinputlisting}

Результаты замеров приведены в таблице \ref{tbl:time}. На рисунках \ref{plt:time_even} и \ref{plt:time_odd} приведены графики зависимостей времени работы алгоритмов от нечетных и четных размеров матриц.

\begin{table}[h]
	\begin{center}
		\caption{Замер времени для разных размеров матриц}
		\label{tbl:time}
		%x*c|
		\begin{tabular}{|c|c|c|c|}
			\hline
			                      & \multicolumn{3}{c|}{\bfseries Время, нс}                                    \\ \cline{2-4}
			\bfseries Размер матрицы & \bfseries Classic & \bfseries Winograd & \bfseries ImpWinograd 
			\csvreader{inc/csv/time.csv}{}
			{\\\hline \csvcoli&\csvcolii&\csvcoliii&\csvcoliv}
			\\\hline
		\end{tabular}
	\end{center}
\end{table}

\begin{figure}[h]
	\centering
	\begin{tikzpicture}
		\begin{axis}[
			axis lines=left,
			xlabel=Размер матрицы,
			ylabel={Время, нс},
			legend pos=north west,
			ymajorgrids=true
		]
			\addplot table[x=size,y=Classic,col sep=comma] {inc/csv/time_classic_even.csv};
			\addplot table[x=size,y=Winograd,col sep=comma] {inc/csv/time_winograd_even.csv};
			\addplot table[x=size,y=ImpWinograd,col sep=comma] {inc/csv/time_imp_winograd_even.csv};
			\legend{Классический, Виноград, Оптим. Виноград}
		\end{axis}
	\end{tikzpicture}
	\captionsetup{justification=centering}
	\caption{Зависимость времени работы алгоритма умножения матриц от размера матрицы. Четные матрицы.}
	\label{plt:time_even}
\end{figure}

\begin{figure}[h]
	\centering
	\begin{tikzpicture}
		\begin{axis}[
			axis lines=left,
			xlabel=Размер матрицы,
			ylabel={Время, нс},
			legend pos=north west,
			ymajorgrids=true
			]
			\addplot table[x=size,y=Classic,col sep=comma] {inc/csv/time_classic_odd.csv};
			\addplot table[x=size,y=Winograd,col sep=comma] {inc/csv/time_winograd_odd.csv};
			\addplot table[x=size,y=ImpWinograd,col sep=comma] {inc/csv/time_imp_winograd_odd.csv};
			\legend{Классический, Виноград, Оптим. Виноград}
		\end{axis}
	\end{tikzpicture}
	\captionsetup{justification=centering}
	\caption{Зависимость времени работы алгоритма умножения матриц от размера матрицы. Нечетные матрицы.}
	\label{plt:time_odd}
\end{figure}

\section{Производительность алгоритмов}

Производительность и объем выделенной памяти при работе алгоритмов указаны на рисунке \ref{img:memory}.

\boximg{50mm}{memory}{Замеры производительности алгоритмов, выполненные при помощи команды \code{go test -bench . -benchmem}}

\section*{Вывод}

Были протестированы различные алгоритмы умножения матриц. По результатам эксперимента классический алгоритм показывает худшие временные показатели. Алгоритм Винограда и оптимизированный алгоритм Винограда работают немногим хуже на нечетных размерах матриц.