\chapter{Аналитическая часть}
Матрицей A размера $[m \times n]$ называется прямоугольная таблица
чисел, функций или алгебраических выражений, содержащая m строк и n столбцов. Числа m и n определяют размер матрицы.\cite{Beloysov} Если число столбцов в первой матрице совпадает с числом строк во второй, то эти две матрицы можно перемножить. У произведения будет столько же строк, сколько в первой матрице, и столько же столбцов, сколько во второй.

Пусть даны две прямоугольные матрицы А и В размеров $[m \times n]$ и $[n \times k]$ соответственно.  
В результате произведение матриц A и B получим матрицу C размера $[m \times  k]$.


$c_{i,j} = \sum\limits_{r=1}^n a_{i,r}\cdot b_{r,j}$ называется произведением матриц A и B \cite{Beloysov}.


\section{Алгоритм Винограда}
Подход алгоритма Винограда является иллюстрацией общей методологии, начатой в 1979 годах на основе билинейных и трилинейных форм, благодаря которым большинство усовершенствований для умножения матриц были получены \cite{Gall2012}.

Рассмотрим два вектора $V = (v1, v2, v3, v4)$ и $W = (w1, w2, w3, w4)$.  

Их скалярное произведение равно (\ref{formula}) 

\begin{equation} \label{formula}
	V \cdot W=v_1 \cdot w_1 + v_2 \cdot w_2 + v_3 \cdot w_3 + v_4 \cdot w_4
\end{equation}

Равенство (\ref{formula}) можно переписать в виде (\ref{formula2}) 
\begin{equation} \label{formula2}
	V \cdot W=(v_1 + w_2) \cdot (v_2 + w_1) + (v_3 + w_4) \cdot (v_4 + w_3) - v_1 \cdot v_2 - v_3 \cdot v_4 - w_1 \cdot w_2 - w_3 \cdot w_4
\end{equation}

Менее очевидно, что выражение в правой части последнего равенства допускает предварительную обработку: его части можно вычислить заранее и запомнить для каждой строки первой матрицы и для каждого столбца второй. 
Это означает, что над предварительно обработанными элементами нам придется выполнять лишь первые два умножения и последующие пять сложений, а также дополнительно два сложения. 

\section*{Вывод}
Были рассмотрены алгоритмы классического умножения матриц и алгоритм Винограда, основное отличие которых — наличие предварительной обработки, а также количество операций умножения.
