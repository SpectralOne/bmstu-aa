\chapter{Технологическая часть}

В данном разделе приведены требования к программному обеспечению, средства реализации и листинги кода.

\section{Требования к ПО}

К программе предъявляется ряд требований:
\begin{itemize}
	\item корректное умножение матриц;
	\item при матрицах неправильных размеров программа не должна аварийно завершаться.
\end{itemize}

\section{Средства реализации}

В качестве языка программирования для реализации данной лабораторной работы был выбран многопоточный язык GO \cite{golang}. Данный выбор обусловлен моим желанием расширить свои знания в области применения данного языка. Так же данный язык предоставляет средства тестирования разработанного ПО.

\clearpage

\section{Листинг кода}

В листингах \ref{lst:m_std}--\ref{lst:imp_winograd} приведены реализации алгоритмов умножения матриц.

\begin{lstinputlisting}[
	caption={Классический},
	label={lst:m_std},
	style={go},
	linerange={14-29},
	]{../src/multiplication/multiplication.go}
\end{lstinputlisting}

\begin{lstinputlisting}[
	caption={Виноград},
	label={lst:m_winograd},
	style={go},
	linerange={32-84},
	]{../src/multiplication/multiplication.go}
\end{lstinputlisting}

\begin{lstinputlisting}[
	caption={Оптимизированный Виноград},
	label={lst:imp_winograd},
	style={go},
	linerange={87-144},
	]{../src/multiplication/multiplication.go}
\end{lstinputlisting}

В таблице \ref{tabular:functional_test} приведены функциональные тесты для алгоритмов умножения матриц.


\renewcommand{\arraystretch}{2}
\begin{table}[h]
	\begin{center}
	\caption{Функциональные тесты}
	\label{tabular:functional_test}
		\begin{tabular}{|*3{>{\renewcommand{\arraystretch}{1}}c|}}
			\hline
			\textbf{Матрица 1} & \textbf{Матрица 2} & \textbf{Ожидаемый результат}\\
			\hline
			$\left( \begin{array}{ccc} 1 & 2 & 3  \\ 4 & 5 & 6 \\ 7 & 8 & 9 \end{array}\right)$ & $\left( \begin{array}{ccc} 1 & 0 & 0 \\ 0 & 1 & 0 \\ 0 & 0 & 1 \end{array}\right)$& $\left( \begin{array}{ccc} 1 & 2 & 3 \\ 4 & 5 & 6 \\ 7 & 8 & 9  \end{array}\right)$\\
			\hline
			$\left( \begin{array}{ccc} 1 & 1 & 1  \\ 1 & 1 & 1 \\ 1 & 1 & 1 \end{array}\right)$ & $\left( \begin{array}{ccc} 1 & 1 & 1 \\ 1 & 1 & 1 \\ 1 & 1 & 1 \end{array}\right)$& $\left( \begin{array}{ccc} 3 & 3 & 3 \\ 3 & 3 & 3 \\ 3 & 3 & 3  \end{array}\right)$\\
			\hline

		\end{tabular}
	\end{center}
\end{table}

\section*{Вывод}

Были разработаны и протестированы спроектированные алгоритмы: стандартного умножения матриц, алгоритм Винограда, а также оптимизированный алгоритм Винограда.
