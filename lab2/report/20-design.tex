\chapter{Конструкторская часть}

\section{Схемы алгоритмов}

На рисунке \ref{img:m_std} приведена схема классического алгоритма умножения матриц.

%\section{Схема алгоритма Винограда}

На рисунке \ref{img:m_winograd} приведена схема алгоритма Винограда.

%\section{Схема оптимизированного алгоритма Винограда}

На рисунке \ref{img:imp_winograd} приведена схема оптимизированного алгоритма Винограда.

\newpage
\simg{1}{m_std}{Схема классического алгоритма умножения матриц}

\newpage
\simg{1}{m_winograd}{Схема алгоритма Винограда}

\newpage
\simg{1}{imp_winograd}{Схема оптимизированного алгоритма Винограда}

\section{Трудоемкость алгоритмов}

Введем модель трудоемкости для оценки алгоритмов:
\begin{itemize}
	\item  базовые операции стоимостью 1 — +, -, *, /, =, ==, <=, >=, !=, +=, [], ++, -- получение полей класса;
	\item оценка трудоемкости цикла: Fц = a + N$\cdot$(a + Fтела), где a - условие цикла;
	\item стоимость условного перехода возьмем за 0, стоимость вычисления условия остаётся.
\end{itemize}

\subsection{Классический алгоритм}

Рассмотрим трудоемкость классического алгоритма:  

Инициализация матрицы результата: $1 + 1 + n_1(1 + 2 + 1) + 1 = 4n_1 + 3$

Подсчет:\\
$1 + n_1(1 + (1 + m_2(1 + (1 + m_1(1 + (8) + 1) + 1) + 1) + 1) + 1) + 1 = 
n_1(m_2(10m_1 + 4) + 4) + 4) + 2 = 10n_1m_2m_1+ 4n_1m_2 + 4n_1 +2
$

\subsection{Алгоритм Винограда}
Аналогично рассмотрим трудоемкость алгоритма Винограда.  \\

Первый цикл: $\frac{15}{2}n_1m_1 + 5n_1 + 2$ 

Второй цикл: $\frac{15}{2}m_2n_2+ 5m_2 + 2$

Третий цикл: $13n_1m_2m_1 + 12n_1m_2 + 4n_1 + 2$

Условный переход: $\begin{bmatrix}
	2    &&, \text{невыполнение условия}\\
	15n_1m_2 + 4n_1 + 2 &&, \text{выполнение условия}\\
\end{bmatrix} $ \\

Итого: $  13n_1m_2m_1 + \frac{15}{2}n_1m_1 +\frac{15}{2}m_2n_2 + 12n_1m_2 + 5n_1 + 5m_2 + 4n_1 + 6 + \\
\begin{bmatrix}
	2    &&, \text{невыполнение условия}\\
	15n_1m_2 + 4n_1 + 2 &&, \text{выполнение условия}\\
\end{bmatrix} $ \\

\subsection{Оптимизированный алгоритм Винограда}

Аналогично Рассмотрим трудоемкость оптимизированого алгоритма Винограда:\\

Первый цикл: $\frac{11}{2}n_1m_1 + 4n_1 + 2$ 

Второй цикл: $\frac{11}{2}m_2n_2+ 4m_2 + 2$

Третий цикл: $\frac{17}{2}n_1m_2m_1 + 9n_1m_2 + 4n_1 + 2$

Условный переход: $\begin{bmatrix}
	1    &&, \text{невыполнение условия}\\
	10n_1m_2 + 4n_1 + 2 &&, \text{выполнение условия}\\
\end{bmatrix} $ \\

Итого: $\frac{17}{2}n_1m_2m_1 + \frac{11}{2}n_1m_1 + \frac{11}{2}m_2n_2 + 9n_1m_2 + 8n_1 + 4m_2 + 6 + \\
\begin{bmatrix}
	1    &&, \text{невыполнение условия}\\
	10n_1m_2 + 4n_1 + 2 &&, \text{выполнение условия}\\
\end{bmatrix} $ \\

\section*{Вывод}

На основе теоретических данных, полученных из аналитического раздела были построены схемы требуемых алгоритмов. Была введена модель оценки трудоемкости алгоритма, были расчитаны трудоемкости алгоритмов в соответсвии с этой моделью.

