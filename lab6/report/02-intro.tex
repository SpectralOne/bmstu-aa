\chapter*{Введение}

\addcontentsline{toc}{chapter}{Введение}

В современной теории алгоритмов существует ряд задач, получение точного ответа в которых возможно только полным перебором всевозможных вариантов множества решений (т.н. NP-полные задачи). Однако зачастую в подобных задачах оптимизации допустимо получение ответа, приближенного к идеальному. В этих целях используются алгоритмы, работающие за приемлемое полиномиальное время, такие как генетические и муравьиные алгоритмы.


Муравьиный алгоритм — один из эффективных полиномиальных алгоритмов для нахождения приближенных решений задачи коммивояжёра, а также решения аналогичных задач поиска маршрутов на графах.

В ходе лабораторной работы предстоит:

\begin{itemize}

	\item рассмотреть муравьиный алгоритм и алгоритм полного перебора в задаче коммивояжера;

	\item реализовать алгоритмы;

	\item сравнить временную эффективность алгоритмов.

\end{itemize}