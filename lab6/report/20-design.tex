\chapter{Конструкторская часть}

\section{Требования к программе}
\textbf{Требования к вводу:}
\begin{itemize}
	\item У ориентированного графа должно быть хотя бы 2 вершины.
\end{itemize}

\textbf{Требования к программе:}
\begin{itemize}
	\item Алгоритм полного перебора должен возвращать кратчайший путь в графе.
\end{itemize}
.  
\newline  
\textbf{Входные данные} - матрица смежности графа.  
\newline
\textbf{Выходные данные} - самый выгодный путь.

\section{Схемы алгоритмов}
В данном разделе будут приведены схемы алгоритмов для решения задачи коммивояжора:
полный перебор \ref{img:perebor} и муравьиный \ref{img:ant}\\

\newpage
\boximg{170mm}{perebor}{Схема алгоритма полного перебора}

\newpage
\boximg{170mm}{ant}{Схема муравьиного алгоритма}

\newpage

\section*{Вывод}
\addcontentsline{toc}{section}{Вывод}
В данном разделе были рассмотрены требования к программе и схемы алгоритмов.
