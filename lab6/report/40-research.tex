\chapter{Исследовательская часть}

\section{Пример работы}

Пример работы программы представлен на рисунке \ref{img:demo}.

\boximg{100mm}{demo}{Демонстрация работы алгоритмов}

\section{Технические характеристики}

Технические характеристики устройства, на котором выполнялось тестирование:

\begin{itemize}
	\item Операционная система: Kali \cite{kali} Linux \cite{linux} 5.9.0-kali1-amd64.
	\item Память: 8 GB.
	\item Процессор: Intel® Core™ i5-8250U \cite{intel} CPU @ 1.60GHz
	\item Количество логических потоков: 8
\end{itemize}

Тестирование проводилось на ноутбуке при включённом режиме производительности. Во время тестирования ноутбук был нагружен только системными процессами.


\section{Время выполнения алгоритмов}

В листинге \ref{lst:test} пример реализации функции тестирования.

\begin{lstinputlisting}[
	caption={Функция тестирования},
	label={lst:test},
	style={go},
	linerange={31-59},
	]{../src/ant/benchmarks.go}
\end{lstinputlisting}

На рисунках \ref{plt:time} приведён график сравнения производительности алгоритмов.

\begin{figure}[!h]
	\centering
	\begin{tikzpicture}
		\begin{axis}[
			axis lines=left,
			xlabel=Количество вершин,
			ylabel={Время, мс},
			legend pos=north west,
			ymajorgrids=true]
			\addplot table[x=count,y=Brute,col sep=comma] {inc/csv/brute.csv};
			\addplot table[x=count,y=Ant,col sep=comma] {inc/csv/ant.csv};
			\legend{Brute, Ant}
		\end{axis}
	\end{tikzpicture}
	\captionsetup{justification=centering}
	\caption{Сравнение алгоритмов.}
	\label{plt:time}
\end{figure}

 
\section*{Вывод}

Была исследована зависимость времени работы реализованных алгоритмов от размера матрицы смежности графа. По результатам эксперимента на малых размерах графа полный перебор значительно выигрывает муравьиных алгоритм в скорости, однако на размера графа больше 8 сложность полного перебора растет очень быстро, а так как муравьиный алгоритм обладает полиноминальной сложностью, он работает быстрее перебора.

