\chapter{Технологическая часть}

В данном разделе приведены требования к программному обеспечению, средства реализации и листинги кода.

\section{Структура ПО}

В данном разделе будет рассмотрена структура ПО \ref{img:idef}.

\simg{0.7}{idef}{Структура ПО}

\section{Средства реализации}

В качестве языка программирования для реализации данной лабораторной работы был выбран многопоточный язык GO \cite{golang}. Данный выбор обусловлен моим желанием расширить свои знания в области применения данного язкыа. Так же данный язык предоставляет средства тестирования разработанного ПО.

Время работы алгоритмов было замерено с помощью функции {\ttfamily Now()} из библиотеки {\ttfamily Time} \cite{lib:time}.

\clearpage

\section{Листинг кода}

В листингах \ref{lst:brute} -- \ref{lst:ant} приведены реализации алгоритма полного перебора всех решений и муравьиного алгоритма.

\begin{lstinputlisting}[
	caption={Алгоритм перебора всех возможных вариантов},
	label={lst:brute},
	style={go},
	linerange={194-245},
	]{../src/ant/ant.go}
\end{lstinputlisting}

\begin{lstinputlisting}[
	caption={Муравьиный алгоритм},
	label={lst:ant},
	style={go},
	linerange={85-192},
	]{../src/ant/ant.go}
\end{lstinputlisting}


\section*{Вывод}

В данном разделе были рассмотрены листинги кода программы и структура ПО.
