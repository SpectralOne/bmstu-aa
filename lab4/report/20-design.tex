\chapter{Конструкторская часть}

\section{Схемы алгоритмов}


Рассмотрим алгоритм Винограда и способы его расспаралеливания.

\section{Функциональная модель}

На рисунке \ref{img:idef0} представлена функциональная модель IDEF0 уровня 1.

\simg{0.6}{idef0}{Функциональная модель IDEF0 уровня 1}

\section{Схемы алгоритмов}

На рисунке \ref{img:m_std} изображена схема классического
алгоритма.

\simg{1}{m_std}{Схема классического алгоритма}

\newpage

На рисунке \ref{img:modvinograd-thread} изображена схема
алгоритма классического умножения с возможность распараллеливания вычислений.

\simg{0.8}{m_paral}{Схема классического алгоритма с возможностью распараллеливания}
\newpage


Распараллеливание вычислений реализовано благодаря добавлению двух новых переменных {\ttfamily left} и {\ttfamily right}, которые указывают на диапазон строк, которые необходимо рассчитать.

\newpage
\section*{Вывод}

Благодаря возможности вычислять каждый элемент результативной матрицы отдельно друг от друга, удалось разделить вычисления по строчкам, выдавая каждому потоку диапазон строк, в которых необходимо считать. После выполнения всех потоков и соответственно расчета всех строк матрицы, получается правильный результат.
