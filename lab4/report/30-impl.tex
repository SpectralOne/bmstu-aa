\chapter{Технологическая часть}

В данном разделе приведены требования к программному обеспечению, средства реализации и листинги кода.

\section{Требования к ПО}

К программе предъявляется ряд требований:
\begin{itemize}
	\item корректное умножение матриц;
	\item при матрицах неправильных размеров программа не должна аварийно завершаться.
\end{itemize}


\section{Средства реализации}

В качестве языка программирования был выбран {\ttfamily Commnon LISP} с компилятором {\ttfamily SBCL}. В данном языке присутствуют нативные потоки.

Для распараллеливания вычислений был использован модуль {\ttfamily lparallel} из библиотеки {\ttfamily quick lisp} \cite{thread}.


\clearpage
\section{Листинг кода}

Исходный код распараллеленного алгоритма приведен в листинге \ref{list:vinth}.

\begin{lstinputlisting}[
	caption={Параллельный классический алгоритм},
	label={list:vinth},
	language={Lisp},
	style={go},
	linerange={1-57},
	]{../src/lab.lisp}
\end{lstinputlisting}

Для тестирования программы были заготовлены следующие тесты в таблице
\ref{table:test}.

\begin{table}[h]
	\caption{Тесты для алгоритмов}
	\label{table:test}
	\centering
	\begin{tabular}{|c|c|c|}
		\hline
		Первая матрца & Вторая матрица & Ожидаемый результат \\
		\hline
		1 2 & 1 2 & \ 7 10 \\
		3 4 & 3 4 & 15 22 \\
		\hline
		1 2 3 & 1 2 3 & \ 30\ \ 36\ \ 42 \\
		4 5 6 & 4 5 6 & \ 66\ \ 81\ \ 96 \\
		7 8 9 & 7 8 9 & 102 126 150 \\
		\hline
		1 2 3 & 1 & 14 \\
		4 5 6 & 2 & 32 \\
		& 3 & \\
		\hline
	\end{tabular}
\end{table}

\section*{Вывод}

Был разработан и протестирован многопоточный алгоритм.

