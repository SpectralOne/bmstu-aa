\chapter{Исследовательская часть}

Проведем тестирование и сравним алгоритмы по времени работы.

\section{Пример работы}

Демонстрация работы программы приведена на рисунке \ref{img:five}.

\boximg{140pt}{five}{Размерность 3x3, 2 потока}

\section{Технические характеристики}

Технические характеристики устройства, на котором выполнялось тестирование:

\begin{itemize}
	\item Операционная система: Kali \cite{kali} Linux \cite{linux} 5.8.10-1kali1 64-bit.
	\item Память: 8 GB.
	\item Процессор: Intel® Core™ i5-8250U \cite{intel} CPU @ 1.60GHz
	\item Количество логических потоков: 8
\end{itemize}

Тестирование проводилось на ноутбуке при включённом режиме производительности. Во время тестирования ноутбук был нагружен только системными процессами.


\section{Результаты тестирования}

Результаты продемонстрированы в таблицах \ref{table:test-res} и \ref{table:test-res-th}.

\begin{table}[h]
	\caption{Результаты однопоточного алгоритма}
	\label{table:test-res}
	\centering
	\begin{tabular}{|c|c|c|}
		\hline
		Первая матрца & Вторая матрица & Результат \\
		\hline
		1 2 & 1 2 & \ 7 10 \\
		3 4 & 3 4 & 15 22 \\
		\hline
		1 2 3 & 1 2 3 & \ 30\ \ 36\ \ 42 \\
		4 5 6 & 4 5 6 & \ 66\ \ 81\ \ 96 \\
		7 8 9 & 7 8 9 & 102 126 150 \\
		\hline
		1 2 3 & 1 & 14 \\
		4 5 6 & 2 & 32 \\
		& 3 & \\
		\hline
	\end{tabular}
\end{table}

\begin{table}[h]
	\caption{Результаты многопоточного алгоритма}
	\label{table:test-res-th}
	\centering
	\begin{tabular}{|c|c|c|}
		\hline
		Первая матрца & Вторая матрица & Результат \\
		\hline
		1 2 & 1 2 & \ 7 10 \\
		3 4 & 3 4 & 15 22 \\
		\hline
		1 2 3 & 1 2 3 & \ 30\ \ 36\ \ 42 \\
		4 5 6 & 4 5 6 & \ 66\ \ 81\ \ 96 \\
		7 8 9 & 7 8 9 & 102 126 150 \\
		\hline
		1 2 3 & 1 & 14 \\
		4 5 6 & 2 & 32 \\
		& 3 & \\
		\hline
	\end{tabular}
\end{table}  

\newpage
\section{Замеры времени}

Время замерялось с помощью макроса {\ttfamily time} \cite{chrono}. 

На рисунке \ref{img:even} представлены результаты замера времени алгоритма.

\begin{figure}[h]
	\begin{tikzpicture}
		\begin{axis}[
			legend pos = north west,
			xlabel=Размерность матрицы,
			ylabel=микросекунды,
			grid = major,
			width = 0.8\paperwidth,
			height = 0.38\paperheight,
			line width = 1
			]
			\legend{
				1 поток,
				2 потока,
				4 потока,
				8 потоков
			};
			
			\addplot coordinates {
				(40, 3402)
				(50, 13563)
				(60, 34506)
				(70, 92528) 
				(80, 177577)
				(90, 335545)
				(100, 638098)
			};
			
			\addplot coordinates {
				(40,  724)
				(50, 3673)
				(60, 10328)
				(70, 26875)
				(80, 65667)
				(90, 125431)
				(100, 346627)
			};
			
			\addplot coordinates {
				(40, 431)
				(50, 1558) 
				(60, 5416)
				(70, 12701)
				(80, 26638)
				(90, 78908)
				(100, 323404)
			};
		
			\addplot coordinates {
				(40, 3204)
				(50, 4361)
				(60, 4671)
				(70, 5608)
				(80, 17556)
				(90, 83576)
				(100, 308048)
			};
			
		\end{axis}
	\end{tikzpicture}
	\caption{Результаты замеров}
	\label{img:even}
\end{figure}

\subsection*{Выводы}

Из графиков отношения размерности матрицы ко времени вычисления видно, что рассчет на одном потоке работает в 2 раза медленнее вычислений на нескольких потоках. Среди распараллеленных вычислений видно, что увеличение числа потоков дает небольшой прирост к скорости примерно на 5\%. Но чем больше потоков используется, тем
этот прирост меньше.
