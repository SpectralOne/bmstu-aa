\chapter{Технологическая часть}

В данном разделе приведены требования к программному обеспечению, средства реализации и листинги кода.

\section{Требования к ПО}

К программе предъявляется ряд требований:
\begin{itemize}
	\item корректная сортировка.
\end{itemize}

\section{Структура ПО}

В данном разделе будет рассмотрена структура ПО \ref{img:idef}.

\simg{0.7}{idef}{Структура ПО}

\section{Средства реализации}

В качестве языка программирования для реализации данной лабораторной работы был выбран многопоточный язык GO \cite{golang}. Данный выбор обусловлен моим желанием расширить свои знания в области применения данного язкыа. Так же данный язык предоставляет средства тестирования разработанного ПО.

Время работы алгоритмов было замерено с помощью функции {\ttfamily Now()} из библиотеки {\ttfamily Time} \cite{lib:time}.

\clearpage

\section{Листинг кода}

В листингах \ref{lst:parallel} -- \ref{lst:linear} приведены реализации параллельного и линейного конвейеров.

\begin{lstinputlisting}[
	caption={Реализация параллельного конвейера},
	label={lst:parallel},
	style={go},
	linerange={27-121},
	]{../src/conv/conv.go}
\end{lstinputlisting}

\begin{lstinputlisting}[
	caption={Реализация реализация линейного конвейера},
	label={lst:linear},
	style={go},
	linerange={123-144},
	]{../src/conv/conv.go}
\end{lstinputlisting}


\section*{Вывод}

В данном разделе была рассмотрена структура ПО и листинги кода программы.
