\chapter{Аналитическая часть}

Конвейер - система поточного производства \cite{mednov}. В терминах программирования ленты конвейера представленны функциями, выполняющими над неким набором данных операции и предающие их на следующую ленту конвейера. Моделирование конвейерной обработки хорошо сочетается с технологией многопоточного программирования - под каждую ленту конвейера выделяется отдельный поток, все потоки работают в ассинхронном режиме.

В качестве предметной области я решил выбрать торты - на первой линии конвейера происходит подготовка, на второй выпекание, на третьей проводят финальные работы по приготовлению.



\section{Параллельное программирование}


При использовании многопроцессорных вычислительных систем с общей памятью обычно предполагается, что имеющиеся в составе системы процессоры обладают равной производительностью, являются равноправными при доступе к общей памяти, и время доступа к памяти является одинаковым (при одновременном доступе нескольких процессоров к одному и тому же элементу памяти очередность и синхронизация доступа обеспечивается на аппаратном уровне). Многопроцессорные системы подобного типа обычно именуются симметричными мультипроцессорами ({\ttfamily symmetric multiprocessors, SMP}).


Перечисленному выше набору предположений удовлетворяют также активно развиваемые в последнее время многоядерные процессоры, в которых каждое ядро представляет практически независимо функциони рующее вычислительное устройство.


Обычный подход при организации вычислений для многопроцессорных вычислительных систем с общей памятью – создание новых параллельных методов на основе обычных последовательных программ, в которых или автоматически компилятором, или непосредственно программистом выделяются участки независимых друг от друга вычислений. Возможности автоматического анализа программ для порождения параллельных вычислений достаточно ограничены, и второй подход является преобладающим. При этом для разработки параллельных программ могут применяться как новые алгоритмические языки, ориентированные на параллельное программирование, так и уже имеющиеся языки, расширенные некоторым набором операторов для параллельных вычислений.


Широко используемый подход состоит и в применении тех или иных библиотек, обеспечивающих определенный программный интерфейс ({\ttfamily API}) для разработки параллельных программ. В рамках такого подхода наиболее известны {\ttfamily Windows Thread API}. Однако первый способ применим только для ОС семейства {\ttfamily Microsoft Windows}, а второй вариант {\ttfamily API} является достаточно трудоемким для использования и имеет низкоуровневый характер.


\section{Организация взаимодействия параллельных потоков}

Потоки исполняются в общем адресном пространстве параллельной программы. Как результат, взаимодействие параллельных потоков можно организовать через использование общих данных, являющихся доступными для всех потоков. Наиболее простая ситуация состоит в использовании общих данных только для чтения. В случае же, когда общие данные могут изменяться несколькими потоками, необходимы специальные усилия для организации правильного взаимодействия.



\section*{Вывод}
Была рассмотрена конвейерная обработка данных, технология параллельного программирования и
организация взаимодействия параллельных потоков.
