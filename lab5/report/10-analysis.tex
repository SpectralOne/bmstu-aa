\chapter{Аналитическая часть}

Конвейер - система поточного производства \cite{mednov}. В терминах программирования ленты конвейера представлены функциями, выполняющими над неким набором данных операции и предающие их на следующую ленту конвейера. Моделирование конвейерной обработки хорошо сочетается с технологией многопоточного программирования - под каждую ленту конвейера выделяется отдельный поток, все потоки работают в асинхронном режиме.

В качестве предметной области было выбрано создание личной карточки пользователя - на первой линии конвейера происходит создание имени, на второй создание почтового адреса, на третьей создание любимого напитка.


\section{Параллельное программирование}


При использовании многопроцессорных вычислительных систем с общей памятью \cite{smpbase} обычно предполагается, что имеющиеся в составе системы процессоры обладают равной производительностью, являются равноправными при доступе к общей памяти, и время доступа к памяти является одинаковым (при одновременном доступе нескольких процессоров к одному и тому же элементу памяти очерёдность и синхронизация доступа обеспечивается на аппаратном уровне). Многопроцессорные системы подобного типа, обычно, именуются симметричными мультипроцессорами ({\ttfamily symmetric multiprocessors, SMP}) \cite{smp}.


Перечисленному выше набору предположений удовлетворяют также активно развиваемые в последнее время многоядерные процессоры \cite{vliw}, в которых каждое ядро представляет практически независимо функционирующее вычислительное устройство.


Обычный подход при организации вычислений для многопроцессорных вычислительных систем с общей памятью – создание новых параллельных методов на основе обычных последовательных программ, в которых или автоматически компилятором, или непосредственно программистом выделяются участки независимых друг от друга вычислений. Возможности автоматического анализа программ для порождения параллельных вычислений достаточно ограничены \cite{anallimits}, и второй подход является преобладающим. При этом для разработки параллельных программ могут применяться как новые <<новые>> языки \cite{f}, поддерживающие параллельное программирование, так и уже имеющиеся языки \cite{golang}, расширенные некоторым набором операторов для параллельных вычислений. Также стоит обратить внимание на готовые пакеты для разработки, сочетающие в себе различные инструменты профилирования, компиляторы и анализаторы \cite{intelXE}.


Параллельное программирование реализуемо либо посредством использования библиотек, обеспечивающих определённый программный интерфейс ({\ttfamily API}) для разработки параллельных программ \cite{lib:cxxthread} либо посредством встроенных в язык интерфейсов \cite{concurrencygo}, если таковые имеются.


\section*{Вывод}
Была рассмотрена конвейерная обработка данных, технология параллельного программирования и организация многопроцессорных вычислительных систем.
