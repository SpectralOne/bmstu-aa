\chapter{Исследовательская часть}

\section{Пример работы}

Лог линейной обработки представлен на рисунке \ref{img:demo_linear}, лог конвейерной обработки представлен на рисунке \ref{img:demo_threads}

% \boximg{25mm}{demo}{Демонстрация работы конвейеров}

\boximg{170mm}{demo_linear}{Лог работы линейной обработки}

\newpage

\boximg{170mm}{demo_threads}{Лог работы конвейерной обработки}

\section{Технические характеристики}

Технические характеристики устройства, на котором выполнялось тестирование:

\begin{itemize}
	\item Операционная система: Kali \cite{kali} Linux \cite{linux} 5.9.0-kali1-amd64.
	\item Память: 8 GB.
	\item Процессор: Intel® Core™ i5-8250U \cite{intel} CPU @ 1.60GHz
	\item Количество логических потоков: 8
\end{itemize}

Тестирование проводилось на ноутбуке при включённом режиме производительности. Во время тестирования ноутбук был нагружен только системными процессами.


\section{Время выполнения алгоритмов}

Алгоритмы тестировались при помощи написания <<бенчмарков>> \cite{gotest}, предоставляемых встроенными в Go средствами. Для такого рода тестирования не нужно самостоятельно указывать количество повторов. В библиотеке для тестирования существует константа $N$, которая динамически варьируется в зависимости от того, был ли получен стабильный результат или нет.

В листинге \ref{lst:test} пример реализации бенчмарка.

\begin{lstinputlisting}[
	caption={Реализация бенчмарка},
	label={lst:test},
	style={go},
	linerange={5-11},
	]{../src/conv/conv_test.go}
\end{lstinputlisting}

На рисунках \ref{plt:time} приведён график сравнения производительности конвейеров.

\begin{figure}[!hb]
	\centering
	\begin{tikzpicture}
		\begin{axis}[
			axis lines=left,
			xlabel=Количество изделий,
			ylabel={Время, нс},
			legend pos=north west,
			ymajorgrids=true]
			\addplot table[x=size,y=Linear,col sep=comma] {inc/csv/linear.csv};
			\addplot table[x=size,y=Parallel,col sep=comma] {inc/csv/parallel.csv};
			\legend{Linear, Parallel}
		\end{axis}
	\end{tikzpicture}
	\captionsetup{justification=centering}
	\caption{Сравнение конвейеров.}
	\label{plt:time}
\end{figure}


\newpage
\section{Производительность алгоритмов}

Производительность и объем выделенной памяти при работе алгоритмов указаны на рисунке \ref{img:mem}.

\boximg{55mm}{mem}{Замеры производительности алгоритмов, выполненные при помощи команды \code{go test -bench . -benchmem}}

\section*{Вывод}

Экперимент показывает, что использование нескольких потоков для реализации конвейерной обработки данных ускоряет алгоритм в несколько раз. При этом возникает ситуация при которой ленты не простаивают. Тратится лишь малое время для передачи данных на линию.
