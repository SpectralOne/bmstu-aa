\chapter{Исследовательская часть}

\section{Пример работы}

Демонстрация работы программы приведена на рисунке \ref{img:levenshtein_demo}.

\boximg{160mm}{levenshtein_demo}{Демонстрация работы алгоритмов нахождения расстояния Левенштейна и Дамерау -- Левенштейна}

\section{Технические характеристики}

Технические характеристики устройства, на котором выполнялось тестирование:

\begin{itemize}
	\item Операционная система: Kali \cite{kali} Linux \cite{linux} 5.8.7-1kali1 64-bit.
	\item Память: 8 GB.
	\item Процессор: Intel® Core™ i5-8250U \cite{intel} CPU @ 1.60GHz
\end{itemize}

Тестирование проводилось на ноутбуке при включённом режиме производительности. Во время тестирования ноутбук был нагружен только системными процессами.

\section{Время выполнения алгоритмов}

Алгоритмы тестировались при помощи написания <<бенчмарков>> \cite{gotest}, предоставляемых встроенными в Go средствами. Для такого рода тестирования не нужно самостоятельно указывать количество повторов. В библиотеке для тестирования существует константа $N$, которая динамически варьируется в зависимости от того, был ли получен стабильный результат или нет.

В листинге \ref{lst:levenshtein_test} пример реализации бенчмарка.

\begin{lstinputlisting}[
	caption={Реализация бенчмарка},
	label={lst:levenshtein_test},
	style={go},
	linerange={1-17},
	]{../src/levenshtein/levenshtein_test.go}
\end{lstinputlisting}

Результаты замеров приведены в таблице \ref{tbl:time}. В данной таблице для значений, для которых тестирование не выполнялось, в поле результата находится NaN.
На рисунках \ref{plt:time_levenshtein} и \ref{plt:time_dl} приведены графики зависимостей времени работы алгоритмов от длины строк.

\begin{table}[h]
	\begin{center}
		\caption{Замер времени для строк, размеромот 5 до 200}
		\label{tbl:time}
		\begin{tabular}{|c|c|c|c|c|}
			\hline
			                      & \multicolumn{4}{c|}{\bfseries Время, нс}                                    \\ \cline{2-5}
			\bfseries Длина строк & \bfseries LevRec & \bfseries RecMat & \bfseries Mat & \bfseries DamLev
			\csvreader{inc/csv/time.csv}{}
			{\\\hline \csvcoli&\csvcolii&\csvcoliii&\csvcoliv&\csvcolv}
			\\\hline
		\end{tabular}
	\end{center}
\end{table}

\begin{figure}[h]
	\centering
	\begin{tikzpicture}
		\begin{axis}[
			axis lines=left,
			xlabel=Длина строк,
			ylabel={Время, нс},
			legend pos=north west,
			ymajorgrids=true
		]
			\addplot table[x=len,y=RecursiveMatrix,col sep=comma] {inc/csv/time_recmat.csv};
			\addplot table[x=len,y=IterativeMatrix,col sep=comma] {inc/csv/time_itmat.csv};
			\legend{Рекурсивный с матрицей, Матричный}
		\end{axis}
	\end{tikzpicture}
	\captionsetup{justification=centering}
	\caption{Зависимость времени работы алгоритма вычисления расстояния Левенштейна от длины строк (рекурсивная с заполнением матрицы и матричная реализации)}
	\label{plt:time_levenshtein}
\end{figure}

\begin{figure}[h]
	\centering
	\begin{tikzpicture}
		\begin{axis}[
			axis lines=left,
			xlabel=Длина строк,
			ylabel={Время, нс},
			legend pos=north west,
			ymajorgrids=true
		]
			\addplot table[x=len,y=IterativeMatrix,col sep=comma] {inc/csv/time_itmat.csv};
			\addplot table[x=len,y=DamerauLevenshtein,col sep=comma] {inc/csv/time_dl.csv};
			\legend{Левенштейн, Д. — Левенштейн}
		\end{axis}
	\end{tikzpicture}
	\captionsetup{justification=centering}
	\caption{Зависимость времени работы матричных реализаций алгоритмов Левенштейна и Дамерау — Левенштейна}
	\label{plt:time_dl}
\end{figure}

\section{Использование памяти}

Алгоритмы Левенштейна и Дамерау — Левенштейна не отличаются друг от друга с точки зрения использования памяти, следовательно, достаточно рассмотреть лишь разницу рекурсивной и матричной реализаций этих алгоритмов.

Максимальная глубина стека вызовов при рекурсивной реализации равна сумме длин входящих строк, соответственно, максимальный расход памяти (\ref{for:99})
\begin{equation}
(\mathcal{C}(S_1) + \mathcal{C}(S_2)) \cdot (2 \cdot \mathcal{C}\mathrm{(string)} + 3 \cdot \mathcal{C}\mathrm{(int)}),
\label{for:99}
\end{equation}
где $\mathcal{C}$ — оператор вычисления размера, $S_1$, $S_2$ — строки, $\mathrm{int}$ — целочисленный тип, $\mathrm{string}$ — строковый тип.

Использование памяти при итеративной реализации теоретически равно
\begin{equation}
(\mathcal{C}(S_1) + 1) \cdot (\mathcal{C}(S_2) + 1) \cdot \mathcal{C}\mathrm{(int)} + 10\cdot \mathcal{C}\mathrm{(int)} + 2 \cdot \mathcal{C}\mathrm{(string)}.
\end{equation}

Выделение памяти при работе алгоритмов указано на рисунке \ref{img:memory}.

\boximg{70mm}{memory}{Замеры производительности алгоритмов, выполненные при помощи команды \code{go test -bench . -benchmem}}

\section*{Вывод}

Рекурсивный алгоритм Левенштейна работает на порядок дольше итеративных реализаций, время его работы увеличивается в геометрической прогрессии. На словах длиной 10 символов, матричная реализация алгоритма Левенштейна превосходит по времени работы рекурсивную в 37000 раз. Рекурсивный алгоритм с заполнением матрицы превосходит простой рекурсивный на аналогичных данных в 12000 раз. Алгоритм Дамерау — Левенштейна по времени выполнения сопоставим с алгоритмом Левенштейна. В нём добавлены дополнительные проверки, и по сути он является алгоритмом другого смыслового уровня.

Но по расходу памяти итеративные алгоритмы проигрывают рекурсивному: максимальный размер используемой памяти в них растёт как произведение длин строк, в то время как у рекурсивного алгоритма — как сумма длин строк.
