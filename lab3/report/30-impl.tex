\chapter{Технологическая часть}

В данном разделе приведены требования к программному обеспечению, средства реализации и листинги кода.

\section{Требования к ПО}

К программе предъявляется ряд требований:
\begin{itemize}
	\item корректная сортировка.
\end{itemize}

\section{Средства реализации}

В качестве языка программирования для реализации данной лабораторной работы был выбран многопоточный язык GO \cite{golang}. Данный выбор обусловлен моим желанием расширить свои знания в области применения данного язкыа. Так же данный язык предоставляет средства тестирования разработанного ПО.

\clearpage

\section{Листинг кода}

В листингах \ref{lst:bubble}--\ref{lst:quick} приведены реализации алгоритмов умножения матриц.

\begin{lstinputlisting}[
	caption={Сортировка пузырьком},
	label={lst:bubble},
	style={go},
	linerange={9-17},
	]{../src/sort/sort.go}
\end{lstinputlisting}

\begin{lstinputlisting}[
	caption={Сортировка вставками},
	label={lst:insert},
	style={go},
	linerange={20-30},
	]{../src/sort/sort.go}
\end{lstinputlisting}

\begin{lstinputlisting}[
	caption={Быстрая сортировка},
	label={lst:quick},
	style={go},
	linerange={45-69},
	]{../src/sort/sort.go}
\end{lstinputlisting}

В таблице \ref{tabular:functional_test} приведены функциональные тесты для алгоритмов сортировок.

\renewcommand{\arraystretch}{2}
\begin{table}[h]
	\begin{center}
	\caption{Функциональные тесты}
	\label{tabular:functional_test}
		\begin{tabular}{|*2{>{\renewcommand{\arraystretch}{1}}c|}}
			\hline
			\textbf{Исходные данные} & \textbf{Ожидаемый результат}\\
			\hline
			1 2 3 4 5 & 1 2 3 4 5\\
			\hline
			5 4 3 2 1 & 1 2 3 4 5\\
			\hline
			2 4 5 3 1 & 1 2  3 4 5\\
			\hline

		\end{tabular}
	\end{center}
\end{table}

\section*{Вывод}

Были разработаны и протестированы спроектированные алгоритмы: сортировки пузырьком, сортировки вставками и быстрой сортировки.
