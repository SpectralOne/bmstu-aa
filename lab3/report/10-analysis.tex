\chapter{Аналитическая часть}

\section{Сортировка пузырьком}

Алгоритм \cite{knut} состоит из повторяющихся проходов по сортируемому массиву. За каждый проход элементы последовательно сравниваются попарно и, если порядок в паре неверный, выполняется обмен элементов.


\section{Сортировка вставками}

На каждом шаге выбирается один из элементов неотсортированной части массива (максимальный/минимальный) \cite{knut} и помещается на нужную позицию в отсортированную часть массива. 


\section{Быстрая сортировка}

Общая идея \cite{quick2} алгоритма состоит в следующем:


\begin{enumerate}

	\item выбрать из массива элемент, называемый опорным. Это может быть любой из элементов массива. От выбора опорного элемента не зависит корректность алгоритма, но в отдельных случаях может сильно зависеть его эффективность;
	\item сравнить все остальные элементы с опорным и переставить их в массиве так, чтобы разбить массив на три непрерывных отрезка, следующих друг за другом: «элементы меньшие опорного», «равные» и «большие»;

	\item для отрезков «меньших» и «больших» значений выполнить рекурсивно ту же последовательность операций, если длина отрезка больше единицы.

\end{enumerate}

На практике массив обычно делят не на три, а на две части: например, «меньшие опорного» и «равные и большие»; такой подход в общем случае эффективнее, так как упрощает алгоритм разделения \cite{quick1}.



\section*{Вывод}
Были рассмотрены три алгоритма сортировки.
